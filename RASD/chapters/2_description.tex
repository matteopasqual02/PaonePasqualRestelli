\chapter{Overall Description}
This section is devoted to an overall description of every part of the system.

\section{Product Perspective}
\subsection{Scenarios}
(forse li ho fatti troppo generici)
\begin{enumerate} 
    \item \textbf{A student creates an account}

    Mario Rossi, a university student eager to participate in an internship, is unsure of how to directly contact companies. 
    
    After learning about Students\&Companies, he decides to explore the platform and create an account. Searching for the site in his browser, Mario arrives at a login page, where he sees an option to create a new account if he is not yet registered. He clicks on this option and follows the account creation steps: entering his email, first name, last name, and date of birth.
    
    Upon completing these basic details, Mario is presented with additional options to enhance his profile, such as uploading a CV and adding a brief personal description. However, he decides to skip this step for now, intending to add these details later, as he is currently focused on just exploring what the platform looks like.
    \item \textbf{A student uploads their CV}

    After exploring the site and deciding he would like to be contacted by companies, Mario decides it’s time to enhance his profile by uploading his CV. To do this, he follows a series of steps:

    First, Mario clicks on his profile to access his personal information. Within his profile, he finds a button labeled "Add your CV" and selects it. This action opens his computer's file browser, where he locates his CV and clicks "Upload." 
    
    Once the file is uploaded, Mario clicks "Publish," making his CV visible to anyone who views his profile.

    This update is also noted by the platform, which analyzes the information within his CV. Based on this analysis, the platform notifies relevant companies who may be searching for profiles similar to Mario's, informing them of the availability of a new candidate.
    \item \textbf{A company advertizes their internship}
    
    TechSolutions, a company seeking interns for a new project, is already familiar with Students\&Companies and has an active account on the platform.
    
    To advertise their open internship position, they navigate to their company profile and select the "Add a new project description" button. This opens a page where they write a detailed description of the job responsibilities and the type of student profile they are seeking. 
    
    Once they have completed the description, they click "Publish," making the internship opportunity visible to all visitors to their profile.

    The platform then analyzes this new project listing and notifies students whose profiles match the requirements for the position. 
    
    
    Additionally, students visiting the profile will now be able to click an "Apply" button next to the project description to submit their applications directly, even if they have not received a notification from the system.
    \item \textbf{A student accepts a recomendation}

    Mario, a student with a profile and CV on Students\&Companies, receives a recommendation email from the platform and a message directly on the site, notifying him of an internship opportunity at a company, TechSolutions, that aligns with his interests.
   
    In this email, Mario clicks on a "See Recommendation" button, which redirects him to his profile. There, he finds a new message containing a link to the company’s account.

    Mario reviews the company’s profile and examines the project that the platform has recommended to him. If he finds it appealing, he can return to the message on his profile and click the "Accept Recommendation" button.

    This action is then flagged by the system to the company, allowing them to contact Mario directly through the platform to coordinate the next steps in the selection process.
    \item \textbf{A company uses the system to manage the interview}

    TechSolutions has established contact with Mario Rossi, a student, via Students\&Companies and begins the preselection process using the tools provided by the platform.

    The first step involves setting up a structured questionnaire through Microsoft Forms, featuring predefined questions designed to help TechSolutions better understand Mario’s interests, skills, and overall suitability for the role.

    Mario completes the questionnaire, and TechSolutions is pleased with his responses. Based on his answers, they decide to move forward with the next stage of the interview process. (Had they found his responses unsatisfactory, they would have notified Mario that he was no longer being considered.)

    TechSolutions then initiates a direct chat with Mario to arrange an interview. The interview can be conducted in person if Mario is able to travel, or via video call if travel is not possible.

    During the interview, the company takes advantage of additional tools offered by Students\&Companies, such as a shared digital whiteboard for collaborative problem-solving, real-time file and document sharing, and access to preloaded questions or skills assessments available within the system. These tools facilitate a more interactive and efficient interview experience, and TechSolutions is pleased with the outcome.

    \item \textbf{A company uses the system to finalize the selection}

    \item \textbf{Students are asked to provide feedback and suggestions after an internship}

    Mario is a student who has participated in an internship found through Students\&Companies. After completing the internship he receives both an email and a message on Students\&Companies.

    The message reads:

    "Subject: Feedback Request on Your Internship Experience with TechSolutions

    Dear Mario,

    We would like to thank you for your participation in the internship with TechSolutions through Students\&Companies. We hope you had a valuable experience and have gained new skills during your time with the company.

    In an effort to continuously improve our platform and the internship process, we would greatly appreciate your feedback. Please take a few moments to fill out a brief form where you can share your thoughts, suggestions, and any areas for improvement. Simply follow this link to access the form: [link].

    Your feedback is essential in helping us refine our services and support both students and companies more effectively.

    Thank you once again for your time and for being a part of Students\&Companies.

    Best regards,
    
    The Students\&Companies Team"

    Mario wishes to provide feedback, so he clicks on the link in the message and fills out the form with his insights.

    \item \textbf{Companies recieve a suggestion on how to make their project description more appealing}

    TechSolutions, a company with an active account on Students\&Companies, has published a project description for an internship position they are looking to fill with a student. After the description is published on their profile, a button remains visible next to it, allowing the company to modify the description at any time.

    A few days later, the company notices that they are not receiving many applications. The system, analyzing the lack of responses, sends both a message and an email to the company with helpful tips and suggestions to improve the visibility and appeal of their internship listing. The system provides specific feedback on their project description, pointing out what may be missing and offering recommendations on how to make it more engaging for students.

    Upon reviewing the message and suggestions, TechSolutions chooses to update and modify their project description to enhance its appeal and attract more applicants.
    \item \textbf{A student makes a complaint}

    

    \item \textbf{A university monitors the situation of an internship}

    \item \textbf{A university handles a complaint}

     \item \textbf{A Student proactively search for an internship (also apply maybe)}

      \item \textbf{A company accepts a recomendation}

     \item \textbf{A student recieves a suggestion on how to make his cv more appealing}

     \item \textbf{A student/company provides feedback about the recomendation process}

     \item \textbf{A student modifies his cv after a suggestion / completation of an internship}
      
\end{enumerate}


\subsection{Class diagram}
a

\subsection{State diagrams}
\begin{enumerate}

\item \textbf{login}

\item \textbf{upload CV}

\item \textbf{Recomendation}

\item \textbf{University handles complaint}
\end{enumerate}
\section{Product Functions}

Here we will include the most important categories of use cases, so the main functions that the system should provide to its users (cose che fa il sistema)

\begin{enumerate}

\item \textbf{Account creation}

\item \textbf{User log in}

\item \textbf{CV uploading}

\item \textbf{Project uploading}

\item \textbf{Recomendation system and statistical analysis}

One of the key features of Students\&Companies is its recommendation system, which acts like a career matching service by connecting students with suitable internship opportunities and informing companies of potential candidates. This system operates through advanced statistical analysis methods, implementing a recommender system that uses data-driven algorithms within its system to optimize the matching process, making it easier for students and companies to connect over mutually beneficial opportunities. (? troppo specifico?)

\item \textbf{Notification of students and companies}

\item \textbf{Interview managing}

\item \textbf{Selection finalization}

\item \textbf{Providing feedback}

\item \textbf{Providing suggestions}

\item \textbf{Execution monitoring}

\item \textbf{Complaining}

\item \textbf{Complaints handling}

\end{enumerate}

\section{User charatteristic}

Università??

There are two types of registered users in S\&C: Students (STs) and Companies (COMs). Each user type has distinct characteristics and roles within the platform:

\begin{itemize}
    \item \textbf{STs}: Students use S\&C to find a company offering internships. To access the platform, they must have a device with an internet connection and an account that includes their email and personal data. Once registered, students can browse available internships, apply for them, and participate in interviews with 
    companies

    \item \textbf{COMs}:  Companies join S\&C to find students suitable for internships. To use the platform, they need a device with an internet connection and an account that includes their email and company information. Through S\&C, companies can view student applications, schedule interviews, and select candidates for internships.
\end{itemize}

Both STs and COMs must register with the platform to access its services, enabling seamless communication between students seeking internships and companies offering opportunities.

\section{Assumptions, dependencies, constraints}
\textbf{[DA1] }Students and companies need to have a device and an internet connection (?)

\textbf{[DA2]} Companies need to have detailed internship descriptions

\textbf{[DA3]} Students need to have a CV

\textbf{[DA4]} Students need to be enrolled at a university 

\textbf{[DA5]} Students need to create an account on S\&C as students.

\textbf{[DA6]} Companies need to create an account on S\&C as Companies.

\textbf{[DA7]} Universities need to create an account on S\&C as Universities.

\textbf{[DA8]} Companies need to be able to conduct an interview

\textbf{[DA9]} Companies need to be able to evaluate an interview

\textbf{[DA10]} Universities need to be informed about a current student's internship

\textbf{[DA11]} Universities need to be able to communicate with Students and Companies
