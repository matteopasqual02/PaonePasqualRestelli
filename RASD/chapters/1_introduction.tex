\chapter{Introduction}

\section{Purpose}
As the demand for skilled interns in various industries continues to rise, providing students with relevant internship opportunities is essential for their professional growth. 
Traditionally, students often struggle to find internships that match their skills, experiences, and career aspirations, while companies face challenges in sourcing qualified candidates. 
Students\&Companies (S\&C) aims to bridge this gap by creating a dynamic platform that helps the matching of university students with companies offering internships tailored to their profiles.

S\&C provides a unified framework for students to search for internships, allowing them to showcase their CVs and preferences. 
Companies can advertise their internship opportunities, detailing the required skills, tasks, and benefits offered.
The platform employs sophisticated recommendation mechanisms, utilizing keyword searches and statistical analyses to enhance the matching process, thereby ensuring a better fit between students and internships.
Furthermore, S\&C allows for the tracking of the interview process and provides constructive suggestions for improving project descriptions and CVs. Additionally, it facilitates the management of complaints from universities, enhancing overall communication and collaboration. 

The main goals of the platform are the following:

\subsection{Goals}

[G1] Companies should be able to advertise the internships they want to offer 

[G2] Students should be able to look for internships 

[G3] Students should be able to be informed about internships that can be interesting  

[G4] Companies should be able to be informed about the availability of a student's CVs 

[G5] Students and Companies should be able to accept a recommendation of a possible match

[G6] Students and Companies should be able to establish contact and participate in an interview

[G7] Students and Companies should be able to provide feedback and suggestions on the provided recommendations

[G8] Students and companies should be able to receive suggestions regarding how to make their submissions (project descriptions for companies and CVs for students)

[G9] Students and companies should be able to keep track of the matchmaking and internship processes

[G10] Students and Companies should be able to complain, communicate problems, and provide information on the status of internships

[G11] Universities should be able to monitor internships

[G12] Universities should be able to handle complaints


\section{Scope}
Students&Companies (S&C) is a platform which aims to connects students looking for internships with companies that offer them. 
Companies can advertize their open internships positions, while students can search for opportunities both in a proactive way, looking through alerts, and a passive way, they are notified by the system when a job offer correspond to their criteria. 
This matchmaking process, in case of acceptance by both the parts involved,  (devo finire)

\subsection{World Phenomena}
[WP1] Students prepare their CVs

[WP2] Students want to take part in an internship experience 

[WP3] Companies want to employ a student as an intern 

[WP4] Companies interview possible candidates 

[WP5] Students inform their university about the internship when it's in the course

[WP6] Companies choose the best candidate

\subsection{Shared Phenomena}

\subsubsection{World-controllled}
[SP1] Students upload their CVs

[SP2] Students go through available internships

[SP3] Companies advertise their internship

[SP3] Students accept the recommendation of an internship

[SP4] Companies accept the recommendation of a student

[SP6] Companies use the system to manage interviews

[SP7] Companies use the system to finalize the selections

[SP8] Universities monitor the situation of the internship

[SP9] Universities handle complaints

[SP10] Students and companies use the system to complain, communicate problems, and provide information about the internship status.

\subsubsection{Machine-controllled}
[SP11] The system informs students when an interesting internship becomes available

[SP12] The system informs companies when an interesting CV becomes available

[SP13] The system asks students and companies to provide feedback and suggestion

[SP14] The system provides suggestions both to companies and students on how to make submissions

[SP15] The system provides a mechanism to monitor the process and internship


\section{Definitions, Acronymous, Abbreviations}
\begin{table}[H]
\centering
\begin{tabular}{|l|l|}
\hline
\textbf{Abbreviation} & \textbf{Description} \\ \hline
RASD & Requirements Analysis \& Specification Document \\ \hline
G* & Goal \\ \hline
WP* & World phenomena \\ \hline
SP* & Shared phenomena \\ \hline
D* & Domain assumption \\ \hline
R* & Functional requirement \\ \hline
UC* & Use case \\ \hline
S\&C & Students \& Companies \\ \hline
ST & Students \\ \hline
COM & Companies \\ \hline
UML & Unified Modelling Language \\ \hline
UI & User Interface \\ \hline
\end{tabular}
\caption{List of Definitions, Acronymous, and Abbreviations}
\label{table:abbreviations}
\end{table}

  
\section{Revision History}

\begin{itemize}
    \item Version 1.0 (04/11/2024)
\end{itemize}

\section{Reference Documents}

The document is based on the following materials:

\begin{itemize}
    \item IEEE Standard Documentation For RASD
    \item The specification of the RASD and DD assignment of the Software Engineering II course a.a. 2024/25 
    \item Slides of the course on WeBeep
\end{itemize}


\section{Document Structure}

\begin{enumerate}
    \item  \textbf{Introduction}: it aims to give a brief description of the project. In particular, it’s focused on the reasons and the goals that are going to be achieved with its development;
 
    \item \textbf{Overall Description}: it is a high-level description of how the system works with a detailed explanation of the phenomena that involve the world, the machine, or both, there is also the domain description with its assumptions;
 
    \item \textbf{Specific Requirements}: in this section, there is a detailed analysis of the requirements needed to achieve the goals. Moreover, it contains more information useful for developers (i.e constraints about HW and SW);
    
    \item \textbf{Formal analysis}: it’s a formal description of the world phenomena using Alloy;

    \item  \textbf{Effort spent}: it shows the time spent to realize this document organized by section;

    \item  \textbf{References}: it contains the references to any documents and software used to write this paper.
    
\end{enumerate}