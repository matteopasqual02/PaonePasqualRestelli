\chapter{Introduction}

\section{Purpose}
As the demand for skilled interns in various industries continues to rise, providing students with relevant internship opportunities is essential for their professional growth. 
Traditionally, students often struggle to find internships that match their skills, experiences, and career aspirations, while companies face challenges in sourcing qualified candidates. 
Students\&Companies (S\&C) aims to bridge this gap by creating a dynamic platform that helps the matching of university students with companies offering internships tailored to their profiles.

S\&C provides a unified framework for students to search for internships, allowing them to showcase their CVs and preferences. 
Companies can advertise their internship opportunities, detailing the required skills, tasks, and benefits offered.
The platform employs sophisticated recommendation mechanisms, utilizing keyword searches and statistical analyses to enhance the matching process, thereby ensuring a better fit between students and internships.
Furthermore, S\&C allows for the tracking of the interview process and provides constructive suggestions for improving project descriptions and CVs. Additionally, it facilitates the management of complaints from universities, enhancing overall communication and collaboration. 

The main goals of the platform are the following:

\subsection{Goals}

[G1] Companies should be able to advertise the interships they offer 

[G2] Students should be able to look for internships

[G3] Students should be able to upload their CV  

[G4] Students should be able to be informed about internships that can be interesting for them 

[G5] Companies should be able to be informed about the availability of a student's CVs 

[G6] Students should be able to accept a recommendation for an internship  

[G7] Companies should be able to accept a recommendation of the availability of a student. 

[G8] A contact must be able to be established between the two parties that have accepted a suitable recommendation  


[G9] Companies must be able to conduct an interview with a student 


[G10] Students must be able to take part in an interview with a company

[G11] Students and companies must be able to provide feedback and suggestions

[G12] Students and companies must be able to receive suggestions regarding how to make their submissions (project descriptions for companies and CVs for students)

[G13] All parties should be able to keep track and monitor the execution and tracking of the matchmaking

[G14] All parties should be able to keep track and monitor internships

[G15] All parties should be able to complain, communicate problems, and provide information on the status of internships

[G16] Universities should be able to monitor internships

[G17] Universities should be able to handle complaints



\section{Scope}
tbd Mati
\subsection{World Phenomena}
[WP] Students prepare their cvs

[WP] Students decide they want to take part in an internship experience

[WP] Companies decide to employ a student as an intern

[WP] Companies decide to use S\&C to advertise their internships

[WP] Companies interview possible candidates (?)

[WP] Students inform their university about the internship when it's in the course


\subsection{Shared Phenomena}

\subsubsection{World-controllled}
[SP] Students go through available internships

[SP] Students decide to accept the recommendation of an internship.

[SP] Companies decide to accept the recommendation of a student. 

[SP] Companies use the system during the interview (?)

[SP] Universities monitor the situation of the internship and handle complaints.

\subsubsection{Machine-controllled}
[SP] The system informs students when an interesting internship becomes available

[SP] The system informs companies when an interesting CV becomes available

[SP] The system asks students and companies to provide feedback and suggestion

[SP] The system provides suggestions both to companies and students on how to make submissions

[SP] The system provides a mechanism to monitor the pross and internship

[SP] The system provides a space to complain, communicate problems, and provide information abouth the internship status.
\section{Definitions, Acronymous, Abbreviations}
\subsubsection{Definitions}
\begin{itemize}
    \item Static analysis tool: Method of debugging that is done by automatically examining the source code without having to execute the program ensuring that is compliant, safe, and secure.
    \item Dynamic analysis tool: Process of testing and evaluating a program thanks to running a test on the code.
\end{itemize}

\subsubsection{Acronymous}
\begin{itemize}
    \item S\&C: Students\&Companies
    \item UML: Unified Modelling Language.
    \item UI: User Interface;
\end{itemize}

\subsubsection{Abbreviations}
\begin{itemize}
    \item G*: goal
    \item WP*: world phenomena
    \item SP*: shared phenomena
    \item D*: Domain assumption
    \item R*: functional requirement
    \item UC*: use case
\end{itemize}
  
\section{Revision History}

\begin{itemize}
    \item Version 1.0 (28/10/2024)
\end{itemize}

\section{Reference Documents}

tbd Matte


\section{Document Structure}

\begin{enumerate}
    \item  Introduction: it aims to give a brief description of the project. In particular, it’s focused on the reasons and the goals that are going to be achieved with its development;
 
    \item Overall Description: it is a high-level description of how the system works with a detailed explanation of the phenomena that involve the world, the machine, or both, there is also the domain description with its assumptions;
 
    \item Specific Requirements: in this section, there is a detailed analysis of the requirements needed to achieve the goals. Moreover, it contains more information useful for developers (i.e constraints about HW and SW);
    
    \item Formal analysis: it’s a formal description of the world phenomena using Alloy;

    \item  Effort spent: it shows the time spent to realize this document organized by section;

    \item  References: it contains the references to any documents and software used to write this paper.
    
\end{enumerate}