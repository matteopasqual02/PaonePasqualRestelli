
\chapter{Specific Requirements}
This section provides a detailed description of the various types of requirements the system must address to achieve all the functionalities outlined. These requirements are essential to ensure the platform operates efficiently, and securely meeting users needs. 

\section{External Interface Requirements}
\subsection{User Interfaces}
The Students\&Companies (S\&C) user interface will be a web app developed to be used by both STs and COMs. It will be accessible to anyone with a device equipped with an internet browser and a reliable internet connection. The platform will provide an intuitive and user-friendly experience, ensuring that users can easily navigate and access its features regardless of their device or operating system.

\subsection{Hardware Interfaces}
The system will be accessible from every device with an Internet Browser to access the website and a reliable Internet connection. The User is free to choose his device like a computer, a tablet, or a smartphone.

The system will be accessible from any device with an internet browser and a reliable internet connection. Users can choose their preferred device, whether it is a computer, tablet, or smartphone. This ensures flexibility and convenience, allowing users to access the platform from anywhere and at any time.

\subsection{Software Interfaces}
The system requires an API to facilitate email sending. These emails can include, for instance, 2FA (two-factor authentication) confirmations or general notifications. This functionality is essential to ensure secure user authentication and to keep users informed about important updates and communications through email.
 
\subsection{Communication Interfaces}
The communication interfaces needed by the system are the HTTPS (Hypertext Transfer Protocol Secure) protocol and the Mail System Transfer Protocol (SMTP). 

\begin{itemize}
    \item \textbf{HTTPS} will be used to ensure secure communication between the client and the server, protecting data integrity and confidentiality during transactions such as login, registration, and other sensitive operations.
    
    \item \textbf{SMTP} will be used for sending emails, enabling the system to handle tasks such as sending account 2FA confirmations, notifications, and other user-related communications efficiently and securely.

\end{itemize}

\section{Functional Requirements}

\textbf{[R1]} The system allows unregistered users to create an account

\textbf{[R2]} The system allows students to upload their CV

\textbf{[R3]} The system allows companies to publish new internships

\textbf{[R4]} The system allows companies to add a description to their internships

\textbf{[R5]} When students want to do a proactive research, the system allows them to go through the available internships

\textbf{[R6]} When doing a search the system allows users to filter internships by a key (?)

\textbf{[R7]} When finding an internship that suits their interests, the system allows students to apply for it. 

\textbf{[R8]} When a new internship that might interest some students becomes avaible, the system notifies them.

\textbf{[R9]} When a student's CV that corresponds to a company's needs becomes available the system informs them. 

\textbf{[R10]} The system allows students to accept a recommendation, applying for that particular internship. 

\textbf{[R11]} The system allows companies to accept a recommendation, inviting the candidate that was proposed.

\textbf{[R12]} The system allows students to accept an invitation of a company for a particular internship, applying for it.

\textbf{[R13]}  When the two parties have accepted a recommendation, or when the company has accepted an application received, the system allows them to establish a contact

\textbf{[R14]} When conducting an interview, the system supports the companies with the interview process

\textbf{[R15]} When conducting an interview, the system supports the companis with the finalization of the selection

\textbf{[R16]} The system allows students and companies to provide feedback and suggestions to feed statistical analysis.

\textbf{[R17] }The system provides suggestions to students regarding how to make their CVs more appealing

\textbf{[R18]} The system provides suggestions to companies regarding how to make their project descriptions more appealing

\textbf{[R19]} During the matchmaking process, the system allows all users to keep track of its execution and outcome

\textbf{[R20]} During the internship the system allows all interested parties to monitor it

\textbf{[R21]} During and ongoing internship, the system allows all users to complain 

\textbf{[R22]} During and ongoing internship, the system allows all users to communicate problems 

\textbf{[R23]} During and ongoing internship, the system allows all users to provide information on its status

\textbf{[R24]} When reports or complaints about the status of an ongoing internship are made, the system allows Universities to see them.

\textbf{[R25]} When complaints about the status of an ongoing internship are made, the system allows Universities to handle them.

\pagebreak

\subsection{Use case diagrams}


\subsection{Use cases}
tabella per ogni uc e poi aggiungi i sequence diagrams
\begin{itemize}
\item Students create an account
\item Companies create an account
\item Student searches and applies for an internship
\item Student recieves an internship recomendation (accept / decline)
\item Company recieves a student’s recomendation (invite students + accept)
\item Interview Process
\item Student receives suggestions on CV
\item Company receives suggestions on Add
\item User leaves feedback
\item User leaves comments on ongoing internship
\item University monitors and handles complaints
\end{itemize}

Come gestiamo i reply messages? mettiamo il ritorno ogni volta che il sistema cambia pagina? (ci vuole la freccia tratteggiata di ritorno ogni volta che facciamo un operazione per segnalare l'azione del sistema in risposta a quell'operazione.)

\textbf{UC1}

\begin{longtable}{|p{0.3\textwidth}|p{0.65\textwidth}|}
\hline
\textbf{Name} & Students creates an account \\
\hline
\textbf{Actor} & Students who want to have access and use the S\&C platform \\
\hline
\textbf{Entry Condition} & 
\begin{itemize}
    \item Student has a valid institutional email
    \item Student has a CV
\end{itemize} \\
\hline
\textbf{Event Flow} & 
\begin{enumerate}
    \item The student opens the platform
    \item The student selects the option of creating a new account
    \item The student inserts his personal attributes: institutional email address, legal name, birth date, and location of residence
    \item The student inserts his CV
    \item The student is assigned a valid username related to his legal name
    \item The student receives a verification email to validate his identity
    \item The student confirms his identity
    \item The student account is created and activated
\end{enumerate} \\
\hline
\textbf{Exit Condition} & The new account is created \\
\hline
\textbf{Exception} & 
\begin{itemize}
    \item The student's email address has already been connected to an existing account. In that case, the system returns an error and redirects the student to the login page.
    \item The student's email address is incorrect or not valid. In this case, the system returns an error, and the flow won't continue until the student inserts a valid email address.
\end{itemize} \\
\hline
\end{longtable}


\textbf{UC\#}

\begin{longtable}{|p{0.3\textwidth}|p{0.65\textwidth}|}
\hline
\textbf{Name} &  Student searchs and applies for an internship\\
\hline
\textbf{Actor} &  Student\\
\hline
\textbf{Entry Condition} &  The student is correctly logged in and has decided he wants to look for an internship. He is on the home page of S\&C.\\
\hline
\textbf{Event Flow} &  
\begin{itemize}
    \item The student clicks on "Available internships" button.
    \item S\&C gives him the correct page.
    \item The student scrolls and reads the list available internships.
    \item If the student finds an internship that interests him, he clicks on its title.
    \item S\&C opens the project description of that internship.
    \item The student reads the description.
    \item If the student is still interested he clicks on the apply button.
    \item S\&C sends the application and the student's profile to the company associated to that internship.
    \item If the student after reading the project description is no longer intersted, he goes back to the previous page by clicking on an arrow button.
    \item S\&C shows him the correct page.
    \item if the student does not find any interesting projects he closes the browser.
\end{itemize} \\
\hline
\textbf{Exit Codition} &  The application is sent and a confirmation message is shown to the student, or the student exits the system.\\
\hline
\textbf{Exception} &  \\
\hline
\end{longtable}


\textbf{UC\#}

\begin{longtable}{|p{0.3\textwidth}|p{0.65\textwidth}|}
\hline
\textbf{Name} &  Student recieves an internship recomendation\\
\hline
\textbf{Actor} &  Student\\
\hline
\textbf{Entry Condition} &  The student has a valid email account and is able to log into the system. The student has uploaded a CV on his profile.\\
\hline
\textbf{Event Flow} &  
\begin{itemize}
\item S\&C finds a match between a student and a company.
\item S\&C sends an email to the student with the recomendation.
\item The student recieves an email stating he has a new recomendation.
\item The student clicks on the button "Go to recomendation".
\item S\&C opens his profile page with the recomendation message.
\item The student reads the project description of the internship recomended, if he is happy with it he clicks on the apply button.
\item S\&C sends a notification to the company that a student recommended by the system has applied.
\item If the student is unhappy with the recomendation he clicks on the button "reject recomendation".
\item S\&C removes the match from the system
\item S\&C notifies the company they matched with that the recomendation has been rejected by the student. 
\end{itemize}
\\
\hline
\textbf{Exit Codition} & The application is sent and a confirmation message is shown to the student, or the student exits the system.\\
\hline
\textbf{Exception} &  
\begin{itemize}
    \item The student does not see the email in time, and when he goes to apply the application window has already closed.
    \item 
\end{itemize}\\
\hline
\end{longtable}



\textbf{UC\#}

\begin{longtable}{|p{0.3\textwidth}|p{0.65\textwidth}|}
\hline
\textbf{Name} &  Company recieves a student's recomendation\\
\hline
\textbf{Actor} &  Company\\
\hline
\textbf{Entry Condition} &  The company has a valid email account and is able to log into the system. The company has uploaded at least one project description on their profile.\\
\hline
\textbf{Event Flow} &  
\begin{itemize}
\item S\&C finds a match between a student and a company.
\item S\&C sends an email to the company with the recomendation.
\item The company recieves an email stating they have a new recomendation.
\item A company's employee clicks on the button "Go to recomendation".
\item S\&C opens their profile page with the recomendation message.
\item The company's employee reads the CV of the student recomended, if they are intrested in the student CV:
\item If the student has already applied they click on the "accept application" button and begin contact with the client.
\item S\&C sends a notification to the student that a company they sent an application to is requesting to contact them.
\item If the student has not already applied there is no "accept application" button, so they click on the "request an application" button and initiate contact with the student.
\item S\&C sends a notification to the student that a company is requesting to contact them.
\item If they are not interested in the student CV:
\item If the student has already applied they click on the button "reject application"
\item S\&C notifies the student they have been rejected.
\item Else, they click on the button "reject recomendation".
\item S\&C removes the match from the system.
\item S\&C notifies the student they matched with that the recomendation has been rejected by the company. 
\end{itemize}\\
\hline
\textbf{Exit Codition} &  Contact is established or the student recieves a notification of the failed match.\\
\hline
\textbf{Exception} &  
\begin{itemize}
    \item 
\end{itemize}
\\
\hline
\end{longtable}



\textbf{UC\#}

\begin{longtable}{|p{0.3\textwidth}|p{0.65\textwidth}|}
\hline
\textbf{Name} &  Interview\\
\hline
\textbf{Actor} &  \\
\hline
\textbf{Entry Condition} &  \\
\hline
\textbf{Event Flow} &  \\
\hline
\textbf{Exit Codition} &  \\
\hline
\textbf{Exception} &  \\
\hline
\end{longtable}



\textbf{UC\#}

\begin{longtable}{|p{0.3\textwidth}|p{0.65\textwidth}|}
\hline
\textbf{Name} &  User leaves feedback\\
\hline
\textbf{Actor} &  User\\
\hline
\textbf{Entry Condition} &  The user has an valid email account. The user has an account on S\&C.\\
\hline
\textbf{Event Flow} &  
\begin{itemize}
    \item 
\end{itemize}\\
\hline
\textbf{Exit Codition} &  \\
\hline
\textbf{Exception} &  \\
\hline
\end{longtable}



\textbf{UC\#}

\begin{longtable}{|p{0.3\textwidth}|p{0.65\textwidth}|}
\hline
\textbf{Name} &  Student recieves suggestion on CV\\
\hline
\textbf{Actor} &  Student\\
\hline
\textbf{Entry Condition} &  The student has an valid email account. The student has an account on S\&C.\\
\hline
\textbf{Event Flow} &  \\
\hline
\textbf{Exit Codition} &  \\
\hline
\textbf{Exception} &  \\
\hline
\end{longtable}

\textbf{UC\#}

\begin{longtable}{|p{0.3\textwidth}|p{0.65\textwidth}|}
\hline
\textbf{Name} &  Company recieves suggestion on project description\\
\hline
\textbf{Actor} &  Company\\
\hline
\textbf{Entry Condition} &  The company has an valid email account. The company has an account on S\&C.\\
\hline
\textbf{Event Flow} &  \\
\hline
\textbf{Exit Codition} &  \\
\hline
\textbf{Exception} &  \\
\hline
\end{longtable}

\begin{longtable}{|p{0.3\textwidth}|p{0.65\textwidth}|}
\hline
\textbf{Name} &  User makes a complaint\\
\hline
\textbf{Actor} &  User\\
\hline
\textbf{Entry Condition} &  The user is logged into their S\&C account.\\
\hline
\textbf{Event Flow} &  
\begin{itemize}
    \item The user goes on their profile page (qui ci va che S\&C gli da la pagina?)
    \item The user clicks on the button "leave a comment" (come vogliamo fare? si puo anche fare che c'è una sezione relativa all'internship in corso)
    \item The user writes a comment or a complaint regarding ??
\end{itemize}
\\
\hline
\textbf{Exit Codition} &  \\
\hline
\textbf{Exception} &  \\
\hline
\end{longtable}

\subsection{Mapping}



\begin{longtable}{|p{0.3\textwidth}|p{0.65\textwidth}|}
\hline
\textbf{Goal} & \textbf{Requirements and Domain Assumptions} \\
\hline
\textbf{[G1]} Companies should be able to advertise the internships they want to offer 
& 
\textbf{Requirements:}
\begin{itemize}
     \item \textbf{[R1]} The system allows unregistered users to create an account
    \item \textbf{[R3]} The system allows companies to publish new internships
    \item \textbf{[R4]} The system allows companies to add a description to their internships
\end{itemize}
\textbf{Domain Assumptions:}
\begin{itemize}
    \item \textbf{[DA1]} Students and companies need to have a device and an internet connection
    \item \textbf{[DA2]} Companies need to have detailed internship descriptions
    \item \textbf{[DA6]} Companies need to create an account on S\&C as Companies.
\end{itemize} \\
\hline
\textbf{[G2]} Students should be able to look for internships 
& 
\textbf{Requirements:}
\begin{itemize}
     \item \textbf{[R1]} The system allows unregistered users to create an account
    \item \textbf{[R2]} The system allows students to upload their CV
    \item \textbf{[R3]} The system allows companies to publish new internships
    \item \textbf{[R5]} When students want to do a proactive research, the system allows them to go through the available internships
    \item \textbf{[R6]} When doing a search the system allows users to filter internships by a key
\end{itemize}
\textbf{Domain Assumptions:}
\begin{itemize}
    \item \textbf{[DA1]} Students and companies need a device and internet connection
    \item \textbf{[DA3]} Students need to have a CV
     \item \textbf{[DA4]} Students need to be enrolled at a university
    \item \textbf{[DA5]} Students need to create an account on S\&C as students.
    \item \textbf{[DA6]} Companies need to create an account on S\&C as Companies.
\end{itemize} \\
\hline
\textbf{[G3]} Students should be able to be informed about internships that can be interesting 
& 
\textbf{Requirements:}
\begin{itemize}
    \item \textbf{[R1]} The system allows unregistered users to create an account
    \item \textbf{[R2]} The system allows students to upload their CV
    \item \textbf{[R3]} The system allows companies to publish new internships
    \item \textbf{[R8]} When a new internship that might interest some students becomes avaible, the system notifies them
\end{itemize}
\textbf{Domain Assumptions:}
\begin{itemize}
      \item \textbf{[DA1]} Students and companies need a device and internet connection
    \item \textbf{[DA3]} Students need to have a CV
     \item \textbf{[DA4]} Students need to be enrolled at a university
    \item \textbf{[DA5]} Students need to create an account on S\&C as students.
    \item \textbf{[DA6]} Companies need to create an account on S\&C as Companies.
\end{itemize} \\
\hline
\textbf{[G4]} Companies should be able to be informed about the availability of a student's CV that its interesting to them
& 
\textbf{Requirements:}
\begin{itemize}
    \item \textbf{[R1]} The system allows unregistered users to create an account
    \item \textbf{[R2]} The system allows students to upload their CV
    \item \textbf{[R3]} The system allows companies to publish new internships
    \item \textbf{[R9]} When a student’s CV that corresponds to a company’s needs becomes available the system informs them.
\end{itemize}
\textbf{Domain Assumptions:}
\begin{itemize}
    \item \textbf{[DA1]} Students and companies need a device and internet connection
    \item \textbf{[DA3]} Students need to have a CV
     \item \textbf{[DA4]} Students need to be enrolled at a university
    \item \textbf{[DA5]} Students need to create an account on S\&C as students.
    \item \textbf{[DA6]} Companies need to create an account on S\&C as Companies.
\end{itemize} \\
\hline
\textbf{[G5]} Students and Companies should be able to accept a recommendation of a possible match
& 
\textbf{Requirements:}
\begin{itemize}
    \item \textbf{[R1]} The system allows unregistered users to create an account
    \item \textbf{[R2]} The system allows students to upload their CV
    \item \textbf{[R3]} The system allows companies to publish new internships
    \item  \textbf{[R8]} When a new intership that might interest some students becomes avaible, the system notifies them
    \item  \textbf{[R9]} When a student’s CV that corresponds to a company’s needs becomes available the system informs them.
    \item  \textbf{[R10]} The system allows students to accept a recommendation, applying for that particular internship.
    \item  \textbf{[R11]} The system allows companies to accept a recommendation, inviting the candidate that was proposed.
\end{itemize}
\textbf{Domain Assumptions:}
\begin{itemize}
     \item \textbf{[DA1]} Students and companies need a device and internet connection
     \item \textbf{[DA3]} Students need to have a CV
     \item \textbf{[DA4]} Students need to be enrolled at a university
    \item \textbf{[DA5]} Students need to create an account on S\&C as students.
    \item \textbf{[DA6]} Companies need to create an account on S\&C as Companies.
\end{itemize} \\
\hline
\hline
\textbf{[G6]}  Students should be able to apply for an internship
& 
\textbf{Requirements:}
\begin{itemize}
    \item \textbf{[R1]} The system allows unregistered users to create an account
    \item \textbf{[R2]} The system allows students to upload their CV
    \item \textbf{[R3]} The system allows companies to publish new internships
    \item \textbf{[R5]} When students want to do a proactive research, the system allows them to go through the available internships
    \item \textbf{[R7]} When finding an internship that suits their interests, the system allows students to apply for it
    \item  \textbf{[R8]} When a new intership that might interest some students becomes avaible, the system notifies them
    \item  \textbf{[R10]} The system allows students to accept a recommendation, applying for that particular internship.
    \item  \textbf{R11]} The system allows companies to accept a recommendation, inviting the candidate that was proposed.
    \item \textbf{[R12]} The system allows students to accept an invitation of a company for a particular internship, applying for it.
\end{itemize}
\textbf{Domain Assumptions:}
\begin{itemize}
     \item \textbf{[DA1]} Students and companies need a device and internet connection
     \item \textbf{[DA3]} Students need to have a CV
     \item \textbf{[DA4]} Students need to be enrolled at a university
    \item \textbf{[DA5]} Students need to create an account on S\&C as students.
    \item \textbf{[DA6]} Companies need to create an account on S\&C as Companies.
\end{itemize} \\
\hline

\textbf{[G7]} Students and Companies should be able to establish contact and participate in an interview 
& 
\textbf{Requirements:}
\begin{itemize}
    \item \textbf{[R1]} The system allows unregistered users to create an account
    \item \textbf{[R2]} The system allows students to upload their CV
    \item \textbf{[R3]} The system allows companies to publish new internships
    \item \textbf{[R7]} When finding an internship that suits their interests, the system allows students to apply for it
    \item  \textbf{[R10]} The system allows students to accept a recommendation, applying for that particular internship.
    \item  \textbf{R11]} The system allows companies to accept a recommendation, inviting the candidate that was proposed.
    \item \textbf{[R12]} The system allows students to accept an invitation of a company for a particular internship, applying for it.
    \item \textbf{[R13]} When the two parties have accepted a recommendation, or when the company has accepted an application received, the system allows them to establish a contact
    \item \textbf{[R14]} When conducting an interview, the system supports the companies with the interview process 
\end{itemize}
\textbf{Domain Assumptions:}
\begin{itemize}
    \item \textbf{[DA1]} Students and companies need a device and internet connection
     \item \textbf{[DA3]} Students need to have a CV
     \item \textbf{[DA4]} Students need to be enrolled at a university
    \item \textbf{[DA5]} Students need to create an account on S\&C as students.
    \item \textbf{[DA6]} Companies need to create an account on S\&C as Companies.
    \item \textbf{[DA8]} Companies need to be able to conduct an interview
\end{itemize} \\
\hline
\textbf{[G8]} Companies should be able to finalize the selection. 
& 
\textbf{Requirements:}
\begin{itemize}
    \item \textbf{[R1]} The system allows unregistered users to create an account
    \item \textbf{[R2]} The system allows students to upload their CV
    \item \textbf{[R3]} The system allows companies to publish new internships
    \item \textbf{[R7]} When finding an internship that suits their interests, the system allows students to apply for it
    \item  \textbf{[R10]} The system allows students to accept a recommendation, applying for that particular internship.
    \item  \textbf{R11]} The system allows companies to accept a recommendation, inviting the candidate that was proposed.
    \item \textbf{[R12]} The system allows students to accept an invitation of a company for a particular internship, applying for it.
    \item \textbf{[R13]} When the two parties have accepted a recommendation, or when the company has accepted an application received, the system allows them to establish a contact
    \item \textbf{[R14]} When conducting an interview, the system supports the companies with the interview process 
    \item \textbf{[R15]} When conducting an interview, the system supports the companis with the finalization of the selection
\end{itemize}
\textbf{Domain Assumptions:}
\begin{itemize}
    \item \textbf{[DA1]} Students and companies need a device and internet connection
    \item \textbf{[DA5]} Companies need an account on S\&C
    \item \textbf{[DA4]} Students need an account on S\&C
    \item \textbf{[DA9]} Companies need to be able to evaluate an interview
\end{itemize} \\
\hline
\textbf{[G8]} Students and Companies should be able to provide feedback and suggestions on the provided recommendations
& 
\textbf{Requirements:}
\begin{itemize}
    \item \textbf{[R13]} The system allows students and companies to provide feedback and suggestions to feed statistical analysis.
\end{itemize}
\textbf{Domain Assumptions:}
\begin{itemize}
    \item \textbf{[DA1]} Students and companies need a device and internet connection
     \item \textbf{[DA3]} Students need to have a CV
     \item \textbf{[DA4]} Students need to be enrolled at a university
    \item \textbf{[DA5]} Students need to create an account on S\&C as students.
    \item \textbf{[DA6]} Companies need to create an account on S\&C as Companies.
\end{itemize} \\
\hline
\textbf{[G10]} Students and companies should be able to receive suggestions regarding how to make their submissions (project descriptions for companies and CVs for students)
& 
\textbf{Requirements:}
\begin{itemize}
    \item \textbf{[R1]} The system allows unregistered users to create an account
    \item \textbf{[R2]} The system allows students to upload their CV
    \item \textbf{[R3]} The system allows companies to publish new internships
    \item \textbf{[R17]} The system provides suggestions to students regarding how to make their CVs more appealing
    \item \textbf{[R18]} The system provides suggestions to companies regarding how to make their project descriptions more appealing
\end{itemize}
\textbf{Domain Assumptions:}
\begin{itemize}
    \item \textbf{[DA1]} Students and companies need a device and internet connection
     \item \textbf{[DA3]} Students need to have a CV
     \item \textbf{[DA4]} Students need to be enrolled at a university
    \item \textbf{[DA5]} Students need to create an account on S\&C as students.
    \item \textbf{[DA6]} Companies need to create an account on S\&C as Companies.
\end{itemize} \\
\hline
\textbf{[G11]} Students and companies should be able to keep track of the matchmaking and internship processes
& 
\textbf{Requirements:}
\begin{itemize}
    \item \textbf{[R1]} The system allows unregistered users to create an account
    \item \textbf{[R2]} The system allows students to upload their CV
    \item \textbf{[R3]} The system allows companies to publish new internships
    \item \textbf{[R13]} When the two parties have accepted a recommendation, or when the company has accepted an application received, the system allows them to establish a contact
    \item \textbf{[R19]} During the matchmaking process, the system allows all users to keep track of its execution and outcome
    \item \textbf{[R20]} During the internship the system allows all interested parties to monitor it
\end{itemize}
\textbf{Domain Assumptions:}
\begin{itemize}
     \item \textbf{[DA1]} Students and companies need a device and internet connection
     \item \textbf{[DA3]} Students need to have a CV
     \item \textbf{[DA4]} Students need to be enrolled at a university
    \item \textbf{[DA5]} Students need to create an account on S\&C as students.
    \item \textbf{[DA6]} Companies need to create an account on S\&C as Companies.
\end{itemize} \\
\hline
\textbf{[G12]} Students and Companies should be able to complain and communicate problems
& 
\textbf{Requirements:}
\begin{itemize}
    \item \textbf{[R1]} The system allows unregistered users to create an account
    \item \textbf{[R21]} During and ongoing internship, the system allows all users to complain
    \item \textbf{[R22]} During and ongoing internship, the system allows all users to communicate problems
    \item \textbf{[R23]} During and ongoing internship, the system allows all users to provide information on its status
\end{itemize}
\textbf{Domain Assumptions:}
\begin{itemize}
    \item \textbf{[DA1]} Students and companies need a device and internet connection
     \item \textbf{[DA3]} Students need to have a CV
     \item \textbf{[DA4]} Students need to be enrolled at a university
    \item \textbf{[DA5]} Students need to create an account on S\&C as students.
    \item \textbf{[DA6]} Companies need to create an account on S\&C as Companies.
\end{itemize} \\
\hline
\textbf{[G13]} Universities should be able to monitor internships
& 
\textbf{Requirements:}
\begin{itemize}
    \item \textbf{[R1]} The system allows unregistered users to create an account
    \item \textbf{[R21]} During and ongoing internship, the system allows all users to complain
    \item \textbf{[R22]} During and ongoing internship, the system allows all users to communicate problems
    \item \textbf{[R23]} During and ongoing internship, the system allows all users to provide information on its status
    \item \textbf{[R24]} When reports or complaints about the status of an ongoing internship are made, the system allows Universities to see them.
\end{itemize}
\textbf{Domain Assumptions:}
\begin{itemize}
    \item \textbf{[DA1]} Students and companies need a device and internet connection
     \item \textbf{[DA3]} Students need to have a CV
     \item \textbf{[DA4]} Students need to be enrolled at a university
    \item \textbf{[DA5]} Students need to create an account on S\&C as students.
    \item \textbf{[DA6]} Companies need to create an account on S\&C as Companies.
    \item \textbf{[DA7]} Universities need to create an account on S\&C as Universities
    \item \textbf{[DA10]} Universities need to be informed about a current student’s internship
\end{itemize} \\
\hline
\textbf{[G14]} Universities should be able to handle complaints
& 
\textbf{Requirements:}
\begin{itemize}
    \item \textbf{[R1]} The system allows unregistered users to create an account
    \item \textbf{[R21]} During and ongoing internship, the system allows all users to complain
    \item \textbf{[R22]} During and ongoing internship, the system allows all users to communicate problems
    \item \textbf{[R23]} During and ongoing internship, the system allows all users to provide information on its status
    \item \textbf{[R24]} When reports or complaints about the status of an ongoing internship are made, the system allows Universities to see them.
    \item \textbf{[R25]} When complaints about the status of an ongoing internship are made, the system allows Universities to handle them.
\end{itemize}
\textbf{Domain Assumptions:}
\begin{itemize}
    \item \textbf{[DA1]} Students and companies need a device and internet connection
     \item \textbf{[DA3]} Students need to have a CV
     \item \textbf{[DA4]} Students need to be enrolled at a university
    \item \textbf{[DA5]} Students need to create an account on S\&C as students.
    \item \textbf{[DA6]} Companies need to create an account on S\&C as Companies.
    \item \textbf{[DA7]} Universities need to create an account on S\&C as Universities
    \item \textbf{[DA10]} Universities need to be informed about a current student’s internshi
    \item \textbf{[DA11]} Universities need to be able to communicate with Students and Companies

\end{itemize} \\
\hline
\end{longtable}

\pagebreak

\section{Performance Requirements}
\begin{itemize}
    \item \textbf{Number of concurrent Users}: According to recent research, websites with similar goals as S\&C have approximately 1.8 million users. Our target is to attract at least 25\% of this user base, which means that S\&C should be capable of handling up to 500,000 concurrent users. This is crucial to ensure the platform operates efficiently and provides a seamless, enjoyable experience for a substantial number of users.

    \item \textbf{Data storage}: The S\&C platform needs to store and manage extensive data related to both STs and COMs. Additionally, it must handle data pertaining to interviews, complaints, issues, data analytics, and other critical information. This requires robust data storage solutions that ensure data integrity, security, and scalability.

    \item \textbf{Time response}: All operations directly executed by S\&C, such as user registration, login, file upload, and evaluation, should have response times within the range of milliseconds. This quick response time is essential to deliver a smooth user experience and maintain user satisfaction. 
   
\end{itemize}


\section{Design Constraints}
\subsection{Standards Compliance}
The S\&C platform is designed to strictly follow several standards to ensure quality, security, and interoperability. 

\begin{itemize}
    \item \textbf{HTTPS Protocol}: The platform implements the HTTPS protocol according to the cryptographic standards established by the Internet Engineering Task Force (IETF), ensuring secure communication between users and the platform.

    \item \textbf{Accessibility Stand}: S\&C complies with the Web Content Accessibility Guidelines (WCAG) to ensure that the platform is accessible to all users, including those with disabilities.

    \item \textbf{Security Standards}: The platform follows security best practices as defined by OWASP (Open Web Application Security Project) and NIST (National Institute of Standards and Technology). This includes password storage encryption using HASH512 + Salt, SSL certificates, and end-to-end communication encryption to protect user data.

    \item \textbf{API Standard}:  The platform uses open standards for API design, such as RESTful APIs, and adheres to specifications like OpenAPI (Swagger) to ensure smooth integration with other systems.

    \item \textbf{Coding Standards}: S\&C follows universally accepted coding guidelines for the primary programming languages used in system development (e.g., Python, Java). This includes adherence to coding conventions such as PEP 8 for Python and Java Coding Conventions for Java.

    \item \textbf{Compliance and Privacy}: The platform complies with privacy regulations such as the General Data Protection Regulation (GDPR) for European citizens, ensuring the protection of user privacy and data rights.
    
\end{itemize}

\subsection{Hardware Limitations}

To access the S\&C platform, both students and companies must have an electronic device, such as a computer, tablet, or smartphone, with a reliable internet connection. 

\begin{itemize}
    \item \textbf{STs}: Students need a device that allows them to access the platform, upload applications, attend interviews, and perform other required activities. They must also have the ability to upload and download files, such as resumes or application documents.

    \item \textbf{COMs}: Companies also need a device with internet access to view applications, schedule interviews, and manage internship postings.
\end{itemize}

Both types of users must have devices that enable them to receive notifications from the platform, ensuring they stay informed about important updates and actions required. The devices should be able to support modern web browsers to access the S\&C platform effectively.


\section{Software System Attributes}
\subsection{Reliability}

The S\&C platform does not manage critical operations. If an operation fails, it can be re-executed without any significant consequences. For example, if the curriculum upload fails, students can simply re-upload it without any issues. Given this non-critical nature, it is reasonable to permit a failure rate of around 1\%, as it does not adversely impact the overall user experience or platform functionality.
 
\subsection{Availability}

The S\&C platform should have high availability, aiming for 24/7 uptime. This is essential to provide continuous access to users without unexpected interruptions, ensuring they can reliably access services whenever needed.

To achieve this, techniques such as load balancing to distribute traffic evenly, failover systems to switch to backup resources during outages, and regular data backups to protect against data loss should be implemented. These measures help maintain seamless operation and ensure that the platform remains robust and dependable at all times.

\subsection{Securuty}

Communication between the user and the S\&C platform is encrypted to avoid data breaches, and unauthorized access, and to ensure the confidentiality and integrity of information shared on the platform.

Furthermore, users must only be able to perform operations that they are authorized to do. For example, a student must not be able to publish an internship, as this function should be restricted to users with specific permissions, such as platform administrators or authorized representatives. Proper access controls and role-based permissions must be implemented to ensure that only authorized users can perform specific actions within the platform
 
\subsection{Maintainability}

The system should be divided into scalable and reusable modules, making it easier to maintain and replace components in case of failure. This modular approach enhances the platform's flexibility and simplifies the process of updating or scaling specific parts without affecting the entire system. 

Ordinary maintenance, including bug fixes and improvements, will be scheduled during nighttime hours when user traffic is minimal to minimize disruption and maintain a smooth user experience. This strategy ensures that the system remains reliable and maintainable while supporting continuous service improvements.

\subsection{Portability}

The S\&C platform does not require any specific hardware or software and must be accessible from any operating system with a modern web browser. This ensures broad compatibility and ease of use for all users. Additionally, a mobile application can be developed to allow users to view the state of battles and other platform activities. Since the mobile app does not require any specialized functions, a non-native approach can be used. This makes it feasible to leverage cross-platform development tools, which can accelerate the development process and reduce the resources needed for maintaining separate codebases for different platforms.